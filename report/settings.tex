\documentclass[a4paper, 10pt]{article}

% Настройки шрифтов
%--------------------------------------
\usepackage[T2A]{fontenc}  % Кодировка шрифтов для русского языка
%--------------------------------------

% Настройки переносов слов
%--------------------------------------
\usepackage{hyphenat}  % Пакет для управления переносами
\hyphenation{ма-те-ма-ти-ка вос-ста-нав-ли-вать}  % Ручные правила переносов
%--------------------------------------

% Настройки языков
%--------------------------------------
\usepackage[english, russian]{babel}  % Поддержка русского и английского языков
%--------------------------------------

% Настройки интерлиньяжа (межстрочного интервала)
%--------------------------------------
\usepackage{setspace}  % Пакет для управления интерлиньяжем
\setstretch{1.3}  % Установка интерлиньяжа в 1.5
%--------------------------------------

% Настройки математики
%--------------------------------------
\usepackage{anyfontsize}  % Пакет для установки произвольного размера шрифта
\usepackage{amsmath, amsfonts, amssymb, amsthm, mathtools}  % Пакеты для математических формул
%--------------------------------------

% Настройки страницы
%--------------------------------------
\usepackage{geometry}  % Пакет для управления геометрией страницы
\geometry{a4paper, portrait, margin=1.5cm, bmargin=1.5cm, tmargin=1.5cm}  % Настройки размеров и полей страницы
%--------------------------------------

% Настройки графики
%--------------------------------------
\usepackage{graphicx}  % Пакет для вставки графики
\usepackage{wrapfig}  % Пакет для обтекания текстом изображений
\usepackage{float}  % Пакет для управления расположением объектов
%--------------------------------------

% Настройки цветов
%--------------------------------------
\usepackage[svgnames]{xcolor}  % Пакет для работы с цветами
\definecolor{codegreen}{rgb}{0,0.6,0}  % Определение цветов для кода
\definecolor{codegray}{rgb}{0.65,0.65,0.65}
\definecolor{codepurple}{rgb}{0.58,0,0.82}
\definecolor{backcolour}{rgb}{0.93, 0.95, 0.96}
\definecolor{keywordcolor}{rgb}{0.23, 0.37, 0.8}
%--------------------------------------

% Настройки для многоколоночного текста
%--------------------------------------
\usepackage{multicol}  % Пакет для создания многоколоночного текста
%--------------------------------------

% Настройки для листинга кода
%--------------------------------------
\usepackage{listings}  % Пакет для вставки и форматирования кода
\lstdefinestyle{mystyle}{  % Определение стиля для листинга кода
    backgroundcolor=\color{backcolour},
    commentstyle=\color{codegreen},
    keywordstyle=\color{keywordcolor}\bf,
    numberstyle=\tiny\color{codegray},
    stringstyle=\color{codepurple},
    basicstyle=\ttfamily\footnotesize,
    breakatwhitespace=false,
    breaklines=true,
    captionpos=b,
    keepspaces=true,
    numbers=left,
    numbersep=5pt,
    showspaces=false,
    showstringspaces=false,
    showtabs=false,
    tabsize=2
}
\lstset{style=mystyle}  % Применение стиля к листингу кода
%--------------------------------------

% Настройки для вставки PDF-файлов
%--------------------------------------
\usepackage{pdfpages}  % Пакет для вставки PDF-файлов
%--------------------------------------

% Настройки для гиперссылок
%--------------------------------------
\usepackage{hyperref}  % Пакет для создания гиперссылок
%--------------------------------------

% Настройки для подчеркивания текста
%--------------------------------------
\usepackage{ulem}  % Пакет для подчеркивания текста
%--------------------------------------

% Настройки для Markdown
%--------------------------------------
\usepackage{markdown}  % Пакет для работы с Markdown
%--------------------------------------

% Настройки для библиографии
%--------------------------------------
\usepackage{biblatex}  % Пакет для работы с библиографией
\addbibresource{literature.bib}  % Указание файла с библиографией
%--------------------------------------

% Настройки для нумерации списков
%--------------------------------------
\renewcommand{\labelenumii}{\arabic{enumi}.\arabic{enumii}}  % Настройка нумерации вложенных списков
\renewcommand{\labelenumiii}{\arabic{enumi}.\arabic{enumii}.\arabic{enumiii}}
\renewcommand{\labelenumiv}{\arabic{enumi}.\arabic{enumii}.\arabic{enumiii}.\arabic{enumiv}}
%--------------------------------------

% Пользовательские команды
%--------------------------------------
\newcommand{\ttgrey}[1]{%  % Команда для выделения текста в серый цвет
  \colorbox{LightGray}{\texttt{#1}}%
}
\newcommand{\bu}[1]{{\color{blue}\uline{#1}}}  % Команда для подчеркивания текста синим цветом
%--------------------------------------

% одинаковые отступы у первого и остальных
\usepackage{indentfirst}
%

% размер формул побольше, все как \[ \]
\everymath{\displaystyle}
%